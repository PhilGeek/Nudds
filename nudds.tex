%!TEX TS-program = xelatex
%!TEX TS-options = -synctex=1
%!TEX encoding = UTF-8 Unicode
%
%  nudds
%
%  Created by Mark Eli Kalderon on 2010-09-22.
%  Copyright (c) 2010. All rights reserved.
%

\documentclass[12pt]{article} 

% Definitions
\newcommand\mykeywords{audition, time, perception} 
\newcommand\myauthor{Mark Eli Kalderon} 
\newcommand\mytitle{Audition, Vision, and Time}

% Packages
\usepackage{geometry} \geometry{a4paper} 
\usepackage{url}
\usepackage{txfonts}
\usepackage{color}
\definecolor{gray}{rgb}{0.459,0.438,0.471}
% \usepackage{setspace}
% \doublespace % Uncomment for doublespacing if necessary
% \usepackage{epigraph} % optional

% XeTeX
\usepackage{fontspec}
\usepackage{xltxtra,xunicode}
\defaultfontfeatures{Scale=MatchLowercase,Mapping=tex-text}
\setmainfont{Hoefler Text}
\setsansfont{Gill Sans}
\setmonofont{Inconsolata}

% Section Formatting
\usepackage[]{titlesec}
\titleformat{\section}[hang]{\fontsize{14}{14}\scshape}{\S{\thesection}}{.5em}{}{}
\titleformat{\subsection}[hang]{\fontsize{12}{12}\scshape}{\S{\thesubsection}}{.5em}{}{}
\titleformat{\subsubsection}[hang]{\fontsize{12}{12}\scshape}{\S{\thesubsubsection}}{.5em}{}{}

% TODO List
% \usepackage{color}
% \usepackage{index} % use index package to create indices
% \newindex{todo}{tod}{tnd}{TODO List} % start todo list
% \newindex{fixme}{fix}{fnd}{FIXME List} % start fixme list
% \newcommand{\todo}[1]{\textcolor{blue}{TODO: #1}\index[todo]{#1}} % macro for todo entries
% \newcommand{\fixme}[1]{\textcolor{red}{FIXME: #1}\index[fixme]{#1}} % macro for fixme entries

% Bibliography
\usepackage[round]{natbib} 

% Title Information
\title{\mytitle} % For thanks comment this line and uncomment the line below
% \title{\mytitle\thanks{}}% 
\author{\myauthor} 
\date{} % Leave blank for no date, comment out for most recent date

% PDF Stuff
% \usepackage[plainpages=false, pdfpagelabels, bookmarksnumbered, backref, pdftitle={\mytitle}, pagebackref, pdfauthor={\myauthor}, pdfkeywords={\mykeywords}, xetex, colorlinks=true, citecolor=gray, linkcolor=gray, urlcolor=gray]{hyperref} 

%%% BEGIN DOCUMENT
\begin{document}

% Title Page
\maketitle
% \begin{abstract} % optional
% \noindent
% \end{abstract} 
\vskip 2em \hrule height 0.4pt \vskip 2em
% Main Content
% \epigraph{}

% Layout Settings
\setlength{\parindent}{1em}

\section{The Neo-Berkelean Framework} % (fold)
\label{sec:the_neo_berkelean_framework}

Matt's discussion of the objects of audition occurs within a neo-Berkelean framework. To bring out the major features of that framework, I want to briefly consider an alternative.

\citet{Berkeley:1734fk} follows Aristotle in maintaining that sound is the proper object of audition. Berkeley maintains, in addition, that sound is the \emph{only} object of audition. Strictly speaking, we hear sounds and not their material sources. In part he argues for this by distinguishing sounds from their sources. Sounds have acoustical properties that their sources lack, and insofar as material sources lack acoustical properties they are inaudible. 

The neo-Berkelean accepts that sound is the proper object of audition. They accept as well that the acoustical properties of sounds distinguish them from their sources. But they deny that sound is the only object of audition. Material sources that can be perceived by other sensory modalities, such as sight, are also the objects of audition, but only derivatively---we hear the source of the sound by hearing the sound. Berkeley goes too far in denying that we hear material sources. Berkeley mistook sound's being the direct or immediate object of audition for sound's being the sole object of audition. If we allow material sources to be the indirect or mediate objects of audition---in the sense that we hear sources by hearing sounds, then the objects of audition include not only proper sensibles but common sensibles as well.

However, in ``The Origin of the Work of Art'' Heidegger writes:
\begin{quote}
    We never really first perceive a throng of sensations, e.g., tones and noises, in the appearance of things \ldots; rather we hear the storm whistling in the chimney, we hear the three-motored plane, we hear the Mercedes in immediate distinction from the Volkswagen. Much closer to us than all sensations are the things themselves. We hear the door shut in the house and never hear acoustical sensations or even mere sounds. \citep[151--152]{Heidegger:1935uq}
\end{quote}
Nothing hangs on Heidegger's apparent acceptance of the empiricist identification of sound with acoustic sensation. What is important is Heidegger's denial of the central neo-Berkelean claim, that we hear the source of sound by hearing the sound. Rather, we hear the source of sound directly.

In undergoing an auditory experience, the material source of a sound is directly or immediately present in that experience. When we attend to our auditory experience, as Heidegger invites us to, we attend to the material sources of sounds and rarely, if at all, to the sounds in distinction from their sources. That is consistent with maintaining that hearing a material source necessarily involves accoustical sensation. And yet Heidegger is clearly denying the neo-Berkelean claim that he hear the source of a sound by hearing the sound. That is a negative result about how to characterize acoustical indirection: There is more to hearing a source by hearing its sound than the necessary accompaniment of the former by the latter.

Heidegger exaggerates when he claims that we never hear acoustical sensations or mere sounds. For he goes on to maintain that we can manage to hear sounds in distinction from their sources only by adopting the aural equivalent of the painterly attitude:
\begin{quote}
    In order to hear a bare sound we have to listen away from things, divert our ears from them, i.e., listen abstractly. \citep[152]{Heidegger:1935uq}
\end{quote}
We can get a sense of how difficult it is to adopt this attitude by considering Pierre Schaeffer's piece ``Étude aux chemins de fer'' (Railroad Study). Whereas traditional composition begins with an abstraction, the score, which is made concrete in playing it, \emph{musique concrète} begins with concrete sounds and abstracts them into a composition through tape looping and sound collage. Yet despite these distancing techniques, the material sources never completely fade from the perceived soundscape. We get a sense of the train's speed, its size, the space surrounding the tracks as well as the space of the interior given the character of the resonance. Indeed, at the end of his career, Schaeffer pronounced \emph{musique concrète} a failure, claiming, perhaps ironically, to have wasted his life. Heidegger's observation was the principle obstacle---it is very difficult to hear bare sounds, to hear sounds without also hearing their sources. And so there are limits to the degree of abstraction that can be achieved with \emph{musique concrète}.

The Berkelean alternative raises an explanatory challenge to the neo-Berkelean---\-to explain how we can experience a source by experiencing its sound. The Heideggerian alternative is a challenge to the very possibility of such an explanation. At the very least, in undergoing an auditory experience, we do not attend to sources by attending to sounds---according to Heidegger, in normal cases, there is no sound that we are attending to. A neo-Berkelean cannot afford to be as sanguine about the Heideggerean alternative as they may be tempted to be about the Berkelean alternative. A promissory note is worth nothing in the face of an inability to repay.

% section the_neo_berkelean_framework (end)

\section{The Objects of Vision and Audition} % (fold)
\label{sec:the_objects_of_vision_and_audition}

Vindicating the neo-Berkelean program is not Matt's aim. Rather his focus is on phenomenological contrasts between audition and vision. This, in turn, is meant to support the claim that we hear material objects by hearing things happening to them.

The phenomenological contrasts Matt draws closely parallel the contrasts that inaugurate C.D. Broad's \citeyearpar{Broad:1965dq} ``Elementary reflections on sense-perception''. % \citet[30]{Broad:1965dq} considers different forms of sense perception from a ``purely phenomenological point of view'':
% \begin{quote}
% 	By this I mean that I shall try to describe them as they appear to any unsophisticated percipient, and as they inevitably \emph{go on appearing} even to sophisticated percipients whose knowledge of the physical and physiological processes involved assures them that the appearances are largely misleading. \citep[30]{Broad:1965dq}
% \end{quote}
In a notorious passage, Broad summarizes these as follows:
\begin{quote}
	In its purely phenomenological aspect \emph{seeing} is ostensibly \emph{saltatory}. It seems to leap the spatial gap between the percipient's body and a remote region of space. Then again, it is ostensibly \emph{prehensive} of the surfaces of distant bodies as coloured and extended, and of external events as colour-occurrences \emph{localized} in remote regions of space. In its purely phenomenological aspect \emph{hearing} is ostensibly prehensive, not of bodies, but only of events or processes as occurrences of sound qualities. It is not ostensibly saltatory, for these events or processes are not heard as localized in remote restricted regions of space. They are heard rather as emanating from remote centres and pervading with diminishing intensity the surrounding space. \citep[32]{Broad:1965dq}
\end{quote}

One contrast, then, between the objects of vision and audition is temporal. Whe\-re\-as the objects of vision include bodies, events, and processes, the objects of audition include only events and processes. The objects of audition are essentially temporally extended. However, Broad follows tradition in emphasizing the temporally extended character of \emph{sounds}. His example of an event that is heard is the tolling of a bell and his example of a process that is heard is the roaring of a waterfall. On the other hand, Matt emphasizes the temporally extended character of, if not sounds, then events or processes involving the sources of sound. Within a neo-Berkelean framework there is no tension here. Indeed the temporally extended character of sound may play a role in explaining how we hear events and processes involving sources of sound. 

Another contrast between the objects of vision and audition is spatial. Matt argues that spatial differences are not fundamental. He contrasts hearing a source and seeing a source where the spatial content of the seen source is restricted to what could figure in audition. Both experiences have the same spatial content, but a phenomenological difference remains. Broad, on the other hand, argues that temporal differences are not fundamental. He contrasts seeing a flash with hearing a sound. Both experiences take events as their objects, but a phenomenological difference remains:
\begin{quote}
	\ldots\ the noise is not literally heard as the occurrence of a certain sound-quality within a limited region remote from the percipient's body. It is certainly not heard as having any shape and size. It seems to be heard as \emph{coming} to one from a certain direction, and it seems to be thought of as pervading with various degrees of intensity the whole of an indefinitely large region surrounding the centre from which it emanates. \citep[32]{Broad:1965dq}
\end{quote}
That a parallel argument can be given for a contradictory conclusion is a basis for doubting Matt's argument here.

One reply would be to claim that events and processes are presented differently in vision and audition. Consider the explanation that Matt noncomittally describes as ``tempting'': We see events and processes by seeing successive states of objects. The presentation of events and processes in a vision would depend on and derive from the presentation of objects and their states. But neither objects nor states unfold in time. As Matt observes the corresponding explanation for audition is unavailable. Hearing an event or process cannot be explained in terms of hearing an object or its state. 

While some have been tempted by this explanation, I am not among them. Not all events or processes that we see involve changes to a substratum. Nietzsche's \citeyearpar[]{Nietzsche1887On-the-Genealog} example is a flash of lighting---there is no lighting that flashes, just the flash. Flashes are unusual events. They are colored whereas most events aren't colored despite having colored participants. (What color was the battle of Kosovo?) But what is presently important is Nietzsche's insight that they also lack substratum. So the presentation of a flash in visual experience is not explainable in terms of the presentation of successive state of a substratum since there is none. Even in the best cases where the event is a change in a state of an object, there are circumstances where we can see the event clearly, and the object less so, if at all. Hiking through the mountains something bounds across your path. Was it a falling rock? An animal? You had a clear view of the object's motion, you could attend to its path, but the object itself was moving too quickly to be seen clearly. You saw neither its size nor shape. It's implausible to claim that you saw the object's motion by seeing the object and its successive position.

The temporal difference may yet be fundamental. However, an emphasis on the essentially temporal character of the objects of audition puts pressure on the idea that we hear material objects at all. Broad writes:
\begin{quote}
    It is about equally common to speak of hearing a \emph{body} and of hearing a \emph{sound}. Thus, e.g., one can say: ``I hear Big Ben'' and ``I hear a series of booming noises.'' Now in this case even the plainest of plain men would admit, with very little pressure, that when he says ``I am hearing Big Ben'' this is short for what would be more fully expressed by saying ``I am hearing Big Ben \emph{striking}.''
\end{quote}
Matt can agree that ``hearing Big Ben'' would be more fully expressed by ``hearing Big Ben striking'' since hearing Big Ben striking is a way of presenting Big Ben in audition. However, Broad's eliminativism is clear---we don't hear Big Ben, only Big Ben striking. Moreover, the essentially temporal character of the objects of audition would provide Broad with prima facie plausible grounds for this denial---Big Ben does not unfold in time the way its sound and its striking do. Broad could concede that we hear Big Ben in hearing Big Ben striking. But that seems more like auditory depiction than auditory perception. Compare Wollheim on visual depiction. Just as we see the depicted scene in the seeing the surface of the picture, we hear Big Ben in hearing Big Ben striking. We hear a dynamic aural image of an object and not the object. Compare seeing, not Big Ben, but its reflection. Broad's challenge for Matt, then, is to explain why objects are indirectly presented in audition as opposed to being merely aurally depicted.

 % A hybrid position is intermediary. The position is hybrid in that experience both presents and represents things. On the hybrid intermediary, only things that unfold in time like events and processes are present in experience, but it is partly by means of these that audition represents material sources. It is intermediary in rsisting Broad's eliminativism but denies that sounds and their soures are differently presented---the former is presented, the latter 

% section the_objects_of_vision_and_audition (end)

% Bibligography
\bibliographystyle{plainnat} 
\bibliography{Philosophy} 

\end{document}