%!TEX TS-program = xelatex
%!TEX TS-options = -synctex=1
%!TEX encoding = UTF-8 Unicode
%
%  nudds
%
%  Created by Mark Eli Kalderon on 2010-09-22.
%  Copyright (c) 2010. All rights reserved.
%

\documentclass[12pt]{article} 

% Definitions
\newcommand\mykeywords{audition, time, perception} 
\newcommand\myauthor{Mark Eli Kalderon} 
\newcommand\mytitle{Audition, Vision, and Time}

% Packages
\usepackage{geometry} \geometry{a4paper} 
\usepackage{url}
\usepackage{txfonts}
\usepackage{color}
\definecolor{gray}{rgb}{0.459,0.438,0.471}
% \usepackage{setspace}
% \doublespace % Uncomment for doublespacing if necessary
% \usepackage{epigraph} % optional

% XeTeX
\usepackage{fontspec}
\usepackage{xltxtra,xunicode}
\defaultfontfeatures{Scale=MatchLowercase,Mapping=tex-text}
\setmainfont{Hoefler Text}
\setsansfont{Gill Sans}
\setmonofont{Inconsolata}

% Section Formatting
\usepackage[]{titlesec}
\titleformat{\section}[hang]{\fontsize{14}{14}\scshape}{\S{\thesection}}{.5em}{}{}
\titleformat{\subsection}[hang]{\fontsize{12}{12}\scshape}{\S{\thesubsection}}{.5em}{}{}
\titleformat{\subsubsection}[hang]{\fontsize{12}{12}\scshape}{\S{\thesubsubsection}}{.5em}{}{}

% TODO List
% \usepackage{color}
% \usepackage{index} % use index package to create indices
% \newindex{todo}{tod}{tnd}{TODO List} % start todo list
% \newindex{fixme}{fix}{fnd}{FIXME List} % start fixme list
% \newcommand{\todo}[1]{\textcolor{blue}{TODO: #1}\index[todo]{#1}} % macro for todo entries
% \newcommand{\fixme}[1]{\textcolor{red}{FIXME: #1}\index[fixme]{#1}} % macro for fixme entries

% Bibliography
\usepackage[round]{natbib} 

% Title Information
\title{\mytitle} % For thanks comment this line and uncomment the line below
% \title{\mytitle\thanks{}}% 
\author{\myauthor} 
% \date{} % Leave blank for no date, comment out for most recent date

% PDF Stuff
\usepackage[plainpages=false, pdfpagelabels, bookmarksnumbered, backref, pdftitle={\mytitle}, pagebackref, pdfauthor={\myauthor}, pdfkeywords={\mykeywords}, xetex, colorlinks=true, citecolor=gray, linkcolor=gray, urlcolor=gray]{hyperref} 

%%% BEGIN DOCUMENT
\begin{document}

% Title Page
\maketitle
% \begin{abstract} % optional
% \noindent
% \end{abstract} 
\vskip 2em \hrule height 0.4pt \vskip 2em
% Main Content
% \epigraph{}

% Layout Settings
\setlength{\parindent}{1em}

\section{The Neo-Berkelean Framework} % (fold)
\label{sec:the_neo_berkelean_framework}

Matt's paper concerns the objects of audition. His discussion of the objects of audition occurs within a neo-Berkelean framework. To bring out the major features of that framework, I want to briefly consider an alternative.

\citet{Berkeley:1734fk} follows Aristotle in maintaining that sound is the proper object of audition. Berkeley maintains, in addition, that sound is the \emph{only} object of audition. Strictly speaking, we hear sounds and not their material sources. In part he argues for this by distinguishing sounds from their sources. Sounds have acoustical properties that their sources lack, and insofar as material sources lack acoustical properties they are inaudible. 

The neo-Berkelean accepts that sound is the proper object of audition, but he denies that sound is the only object of audition. Material sources which can be perceived by other sensory modalities, such as sight, are also the objects of audition, but only derivatively---we hear the source of the sound by hearing the sound. Berkeley goes too far in denying that we hear material sources. Berkeley mistook sound's being the direct object of audition for sound's being the sole object of audition. If we allow material sources to be the indirect objects of audition---in the sense that we hear the sources of sounds by hearing sounds, then the objects of audition include not only proper sensibles but common sensibles as well.

In ``The Origin of the Work of Art'' Heidegger writes:
\begin{quote}
    We never really first perceive a throng of sensations, e.g., tones and noises, in the appearance of things \ldots; rather we hear the storm whistling in the chimney, we hear the three-motored plane, we hear the Mercedes in immediate distinction from the Volkswagen. Much closer to us than all sensations are the things themselves. We hear the door shut in the house and never hear acoustical sensations or even mere sounds. \citep[151--152]{Heidegger:1935uq}
\end{quote}
Nothing hangs on Heidegger's apparent acceptance of the empiricist identification of sound with acoustic sensation. What is important is Heidegger's denial of the central neo-Berkelean claim, that we hear the source of sound by hearing the sound. Rather, we hear the source of sound directly.

In undergoing an auditory experience, the material source of a sound is directly or immediately present in that experience. When we attend to our auditory experience, as Heidegger invites us to, we attend to the material sources of sounds and rarely, if at all, to the sounds in distinction from their sources. That is consistent with maintaining that hearing a material source necessarily involves accoustical sensation. And yet Heidegger is clearly denying the neo-Berkelean claim that he hear the source of a sound by hearing the sound. That is a negative result about how to characterize acoustical indirection, there is more to hearing a source by hearing its sound than necessary accompaniment of the former by the latter.

Heidegger exaggerates when he claims that we never hear acoustical sensations or mere sounds. For he goes on to maintain that we can manage to hear sounds in distinction from their sources only by adopting the aural equivalent of the painterly attitude:
\begin{quote}
    In order to hear a bare sound we have to listen away from things, divert our ears from them, i.e., listen abstractly. \citep[152]{Heidegger:1935uq}
\end{quote}
We can get a sense of how difficult it is to adopt this attitude by considering Pierre Schaeffer's piece ``Étude aux chemins de fer'' (Railroad Study). Whereas traditional composition begins with an abstraction, the score, which is made concrete in playing it, \emph{musique concrète} begins with concrete sounds and abstracts them into a composition through tape looping and sound collage. Yet despite these distancing techniques the material sources never completely fade from the perceived soundscape. We get a sense of the train's speed, its size, the space surrounding the tracks as well as the space of the interior given the character of the resonance. Indeed, at the end of his career, Schaeffer pronounced \emph{musique concrète} a failure, claiming to have wasted his life. Heidegger's observation was the principle obstacle---it is very difficult to hear bare sounds, to hear sounds without also hearing their sources. And so there are limits to the degree of abstraction that can be achieved with \emph{musique concrète}.

The Berkelean alternative raises an explanatory challenge to the neo-Berkelean---to explain how we can experience the source of the sound by experiencing the sound. The Heideggerian alternative is a challenge to the very possibility of such an explanation. At the very least, in undergoing an auditory experience, we do not attend to the sources of the sound present in experience by attending to the sound---according to Heidegger, in normal cases, there is no sound that we are attending to. A neo-Berkelean cannot afford to be as sanguine about the Heideggerean alternative as he may be tempted to be about the Berkelean alternative. A promissory note is worth nothing in the face of evidence of inability to repay.

% section the_neo_berkelean_framework (end)

\section{The Objects of Vision and Audition} % (fold)
\label{sec:the_objects_of_vision_and_audition}

Executing the neo-Berkelean program is not Matt's aim. Rather his focus is on the temporally extended character of the objects of audition and how this supports his claim that we hear material objects by hearing things happening to them. 

% section the_objects_of_vision_and_audition (end)

% Bibligography
\bibliographystyle{plainnat} 
\bibliography{Philosophy} 

\end{document}